\documentclass{automatextcc}

% Caminhos da Pasta das Figuras
\graphicspath{{figuras/}}

%\makeindex  Opcional (Índice Remissivo)


\begin{document}


\title{Compondo Músicas utilizando Redes Neurais Recorrentes}
\author{Nicolas Mathias Hahn}

% orientador(a) do trabalho {nome}{Orientador(a)}
\advisor{Prof. Dr. Guilherme Pumi}{Orientador(a)}
% universidade onde obteve o título e atual
\advisorinfo{Doutor pela Universidade Federal do Rio Grande do Sul, Porto Alegre, RS}{UFRGS}

% banca examinadora:
\examinera{Prof. Dr. xxx}
\examinerainfo{Doutor pela XX -- Cidade, Estado}{Universidade}

% departamento:
\dept{\DEST}

% data de entrega
\date{Outubro de 2022}


% Capa
\maketitulo

% Folha de rosto
\makefolhaderosto

% Folha de aprovação
\makefolhadeaprovacaoA % Um membro na banca
%\makefolhadeaprovacaoB % Dois membros na banca


% Epígrafe (OPCIONAL)
\newpage
\vspace*{\fill}
\begin{flushright} % mexer aqui
	\textit{``Eu tinha um pintinho chamado Relã, toda vez que chovia, Relã piava!''}.
\end{flushright}

% Agradecimentos
\newpage
\chapter*{Agradecimentos}
Agradeço a xxx. Opcional % mexer aqui

% palavras chave
    % português
\keyword{Redes Neurais}
\keyword{Música}
    % inglês
\keyworde{Neural Networks}
\keyworde{Music}

% resumo 
    % português
\begin{abstract}
Este trabalho ....
\end{abstract}
    % inglês
\begin{englishabstract}
In this work ....
\end{englishabstract}

% sumário (Obrigatório)
\tableofcontents

% lista de ilustrações (Obrigatório)
\listoffigures

% lista de tabelas (Obrigatório)
\listoftables


%%%%%%%%%%%%%%%%%%%%%%%%%%%%%%%%%%%
%%%%  Introdução
%%%%%%%%%%%%%%%%%%%%%%%%%%%%%%%%%%%
\chapter{Introdução}

% talvez olhar estes artigo:
% https://musica.ufmg.br/nasnuvens/wp-content/uploads/2020/11/2016-38-A-interatividade-nas-trilhas-sonoras-de-jogos-digitais-e-um-comparativo-com-a-música-de-cinema.pdf
% https://ccrma.stanford.edu/~blackrse/algorithm.html


% referencial teórico
\section{Referencial Teórico}

    % RNN 
\subsection{Redes Neurais Recorrentes - RNN}

% https://en.wikipedia.org/wiki/Recurrent_neural_network
% https://www.ibm.com/cloud/learn/recurrent-neural-networks
% https://matheusfacure.github.io/2017/07/12/activ-func/
\subsubsection{CharRNN}
\subsubsection{Long Short-Term Memory - LSTM}
\subsection{Função de Ativação}
O objetivo da função de ativação é determinar se o neurônio de uma rede neural será ativado ou não. É uma função não linear que tipicamente converte o resultado de um neurônio para um valor dentro do intervalo $[0,1]$ ou $[-1,1]$. Dentre as funções mais comuns, podemos citar três:

\begin{itemize}
    \item sigmóide: $g(x) = \frac{1}{1+e^{-x}}$ 
    % grafico da sigmóide
    \item tangente hiperbólica (tanh): $g(x) = \frac{e^{x}-e^{-x}}{e^{x}+e^{-x}}$
    % grafico da tanh
    \item ReLU: $g(x) = max(0,x)$
    % grafico da ReLU
\end{itemize}

% coletei do site da ibm, mas falta tratar texto e coletar fonte formal

    % notação ABC
\subsection{Notação ABC}

ABC é uma linguagem para notação de música - melodia, letra e cifra - usando caracteres em formato ASCII - \textit{American Standard Code for Information Interchange} (sistema de representação de letras, algarismos e sinais de pontuação e de controle, através de um sinal codificado em forma de código binário - cadeias de bits formada por vários 0 e 1 - desenvolvido a partir de 1960, que representa um conjunto de 128 sinais: 95 sinais gráficos - letras do alfabeto latino, algarismos arábicos, sinais de pontuação e sinais matemáticos - e 33 sinais de controle, utilizando 7 bits para representar todos os seus símbolos.). Desde a introdução ao final de 1991 por Chris Walshaw, se tornou muito popular e existem agora vários programas (para sistemas operacionais diversos, como Windows, MacOS, Unix e mesmo para PDAs) que podem ler notação ABC, convertendo-a em partitura ou tocando-a através de alto-falantes de um computador. Uma característica única de programas ABC é a possibilidade de manipular tanto coletâneas de músicas quanto peças musicais individuais.

\newpage
Exemplo: \newline
X: 308                  \% número da peça (index) \newline
T: Quem não sabe lê     \% título \newline
O: Bahia, capital       \% origem \newline
M: 2/4                  \% metro (compasso) \newline
L: 1/16                 \% unidade de duração \newline
Q: 1/4=84               \% andamento (tempo) \newline
K: A                    \% tom \newline
\% \newline
"A"A2 A2 c3 A | "Bm"B8 | "D"A2 A2 d3 c | "A"B A A A - A A3 | \newline 
w: Quem não sa-be lê Quem não sa-be lê o A-be--cê \newline
\% \newline
"A"A2 A2 e3 c | "Bm"B8 | "D"A2 A2 d3 c | "A"B A A A - A A3 | \newline
w: Ve-nha is-tu-dar Ve-nha is-tu-dar o Be-a--bá


% falta:
% - tratar texto (basicamente ctrl+c e ctrl+v) 
% - inserir imagem/explicacoes do formato

% fonte: https://en.wikipedia.org/wiki/ABC_notation
% falta localizar bibliografia mais formal


%%%%%%%%%%%%%%%%%%%%%%%%%%%%%%%%%%%
%%%%  Metodologia
%%%%%%%%%%%%%%%%%%%%%%%%%%%%%%%%%%%
\chapter{Metodologia}

% coleta de dados
\section{Coleta de Dados}

A coleta de dados foi realizada por meio de um \textit{crawler/bot}, desenvolvido na linguagem Python, para extrair músicas em formato \textit{.abc} do site \href{https://abcnotation.com/}{``abcnotation.com''}. Após uma semana de execução, foram obtidos 184.900 arquivos contendo diversas informações sobre as peças musicais (como título, autor, tonalidade, entre outras).

% tratamento de dados
\section{Tratamento de Dados}

% rede neural
\section{Rede Neural}
\subsection{Arquitetura}
\subsection{Parâmetros de Treinamento}

%%%%%%%%%%%%%%%%%%%%%%%%%%%%%%%%%%%
%%%%  Resultados
%%%%%%%%%%%%%%%%%%%%%%%%%%%%%%%%%%%
\chapter{Resultados}

%%%%%%%%%%%%%%%%%%%%%%%%%%%%%%%%%%%
%%%%  Conclusão
%%%%%%%%%%%%%%%%%%%%%%%%%%%%%%%%%%%
\chapter{Conclusão}

%%%%%%%%%%%%%%%%%%%%%%%%%%%%%%%%%%%
%%%%  Referências
%%%%%%%%%%%%%%%%%%%%%%%%%%%%%%%%%%%
\chapter{Referências}




%%% DAQUI PRA BAIXO É ESTRUTURA ANTIGA A SER DESCARTADA

%%%%%%%%%%%%%%%%%%%%%%%%%%%%%%%%%%%%%%%%%%%%%%%%%%%%%%%%%%%%%%%%%%%%%%%%%%%%%%%%
% COMEÇA O TEXTO DO TRABALHO DE CONCLUSÃO
%%%%%%%%%%%%%%%%%%%%%%%%%%%%%%%%%%%%%%%%%%%%%%%%%%%%%%%%%%%%%%%%%%%%%%%%%%%%%%%%
\chapter{Introdução}

A utilização de músicas é muito presente em filmes, séries, jogos; essas obras, que usualmente fazem parte de uma trilha sonora, têm os mais diversos estilos e servem para inúmeras finalidades (animar o expectador, emocioná-lo ou criar tensão, para citar algumas). No entanto, compor músicas não é algo simples e, muitas vezes, requer um compositor ou grupo de compositores para que seja criada uma nova faixa musical. Apesar disso, técnicas como modelagem sistêmica \citep[veja][]{da2017modelagem} ou machine learning (mais especificamente as técnicas de redes neurais) são exploradas com o intuito de realizar uma composição \citep[ver em][]{agarwala2017music} seja como inspiração para o compositor, seja como uma automatização do processo.

Este trabalho tem o intuito de revisar e estudar alguns dos métodos mais utilizados na literatura para composição musical, principalmente aqueles que requerem apenas arquivos de áudio (músicas ou MIDI - \textit{Musical Instrument Digital Interface}), como machine learning. Por exemplo, é possível desenvolver um modelo estatístico (utilizando redes neurais e técnicas de deep learning) que será treinado com diferentes músicas dos mais diversos estilos e, após isso, predizer (compor) uma nova peça utilizando uma seleção de faixas como entrada (input). Além disso, um dos pontos de partida deste trabalho é o de replicar e de validar o que foi apresentado em \cite{agarwala2017music}, coletar os dados das mesmas fontes citadas, como o \href{https://abcnotation.com/}{``abcnotation.com''} e o \href{https://thesession.org/}{``thesession.org''}, e procurar compor músicas o mais ``naturais'' possíveis, considerando a audição humana. 

Quando escutamos a uma música, nem sempre gostamos do que ouvimos, seja por uma questão de estilo ou pelo nosso momento atual. Independente do motivo, podemos afirmar que o ato de apreciar é subjetivo e pode ser diferente para cada indivíduo. Devido a isso, cada um pode ter percepções diferentes do que imagina ser uma música composta por computador (talvez com um caráter mais ``artificial'') e uma música composta por um ser humano (caráter mais ``natural''). Devido a isso, outro objetivo deste trabalho é o de propor uma pesquisa on-line em que será solicitado aos respondentes se o áudio escutado foi composto por uma máquina ou não. Além disso, serão comparadas também músicas resultantes dos modelos originados pelos métodos estudados e pela nova proposta de método. Por conseguinte, a pesquisa irá responder, idealmente, qual dos modelos mais compôs músicas similares a uma composição humana.



\newpage
\section{Metodologia de Pesquisa}

A presente pesquisa contém um caráter mais teórico do que aplicado, considerando que seu intuito principal é o de reproduzir e verificar os resultados apresentado em \cite{agarwala2017music}, bem como explorar e apresentar uma discussão das ferramentas de deep learning utilizadas para composição musical. Além disso, serão feitas aplicações das técnicas e métodos estudados para a composição de músicas inéditas baseadas em um conjunto personalizado de faixas musicais.

O primeiro passo compreende a coleta das músicas que formarão a base de dados de treino dos modelos. Para tal, utilizaremos sites como o \href{https://abcnotation.com/}{``abcnotation.com''} e o  \href{https://thesession.org/}{``thesession.org''} para a pesquisa (os arquivos vêm codificados no formato ABC, mas serão convertidos para MIDI). Após esse passo, serão realizadas análises descritivas dos dados coletados para se ter uma noção de estilos ou autores das músicas. Por conseguinte, serão treinados modelos utilizando diferentes técnicas de deep learning (como RNN - \textit{Recurrent Neural Networks}, CharRNN e CNN - \textit{Convolutional Neural Networks}) com o objetivo de compor peças musicais com base em uma coleção de músicas pré-selecionadas. Assim sendo, disponibilizaremos as faixas compostas a usuários para avaliarem se, na opinião deles, as músicas foram feitas por máquinas ou por seres humanos.

O desenvolvimento do estudo se dará por meio do software \href{https://www.python.org}{Python}, visto que é bastante utilizado na área de deep learning, além de \cite{agarwala2017music} conter um repositório no \href{https://github.com/yinoue93/CS224N_proj}{GitHub}. Logo, podemos resumir o método de pesquisa em 7 (sete) etapas:

\begin{itemize}
    \item \textbf{Etapa 1 - Revisão Bibliográfica}: estudar artigo base e revisar a literatura dos métodos utilizados para composição musical por meio de modelos.
    \item \textbf{Etapa 2 - Programação dos Algoritmos}: pesquisar, programar e entender como aplicar os métodos pesquisados para atingir o objetivo de composição musical.
    \item \textbf{Etapa 3 - Coleta de Dados}: estudar o código no repositório do artigo base para realizar a coleta dos dados por meio de um robô.
    \item \textbf{Etapa 4 - Desenvolvimento da Pesquisa}: treinar modelos com os diferentes métodos e compor músicas com diferentes conjuntos musicais selecionados.
    \item \textbf{Etapa 5 - Análise dos Resultados}: avaliar se as músicas resultantes foram compostas por máquinas por meio de uma pesquisa de opinião.
    \item \textbf{Etapa 6 - Discussão e Conclusão}: comentários sobre o estado da arte, desempenho dos modelos, obstáculos encontrados e possibilidades de expansão da pesquisa.
    \item \textbf{Etapa 7 - Redação Final da Monografia/Artigo}: polimento final dos textos, bem como organização de um repositório dos códigos utilizados.

\end{itemize}


%%%%%%%%%%%%%%%%%%%%%%%%%%%%%%%%%%%
%%%%  Referencial Teórico
%%%%%%%%%%%%%%%%%%%%%%%%%%%%%%%%%%%

%https://pdf.zlibcdn.com/dtoken/9d711d2ead6feb1b55416a2482f22ed6/Neural_Networks_and_Deep_Learning._A_Textbook_by_C_3583051_(z-lib.org).pdfhttps://pdf.zlibcdn.com/dtoken/9d711d2ead6feb1b55416a2482f22ed6/Neural_Networks_and_Deep_Learning._A_Textbook_by_C_3583051_(z-lib.org).pdf

%https://arxiv.org/pdf/1506.02078)

\newpage
\section{Referencial Teórico}

%\pumi{aqui o ideal é deixar implícito que um dos objetivos do trabalho é revisar o estado da arte no que tange composição artificial de músicas, citar quais são os artigos chave e dizer quais serão os artigos que focarás no trabalho. }

%\subsection{Referencial Teórico}
%\subsection{Estado da Arte}
%\subsection{Seu trabalho no contexto da literatura}

A principal referência utilizada neste trabalho foi o artigo \cite{agarwala2017music}. Demais fontes bibliográficas como \cite{da2017modelagem} e \cite{de2018deep} envolvem diferentes abordagens e técnicas para atingir um objetivo específico: compor músicas. Isso posto, compor músicas sem conhecer a linguagem musical é algo que tem sido cada vez mais explorado, e é algo que podemos observar em \cite{agarwala2017music}.

\subsection{Deep Learning}
Segundo \cite{aggarwal2018DeepLearning}, redes neurais foram desenvolvidas (inicialmente por \cite{neural1943}) para simular o sistema nervoso humano em máquinas para ``aprenderem'' tarefas similares às executadas pelos neurônios do ser humano. No entanto, tal objetivo não é nem um pouco simples de ser alcançado, considerando que o poder de processamento do mais rápido computador é apenas uma fração da capacidade do cérebro humano. Apesar de tal obstáculo, o aumento do poder computacional e o crescente volume de dados disponíveis permitiu que as redes neurais tivessem maior sucesso em aprender tarefas simples para um ser humano, mas complexas para uma máquina (por exemplo, reconhecimento de imagens). Por conseguinte, ainda segundo \cite{aggarwal2018DeepLearning}, esta área foi renascida com o nome de \textit{deep learning}. 


\subsection{Redes Neurais Recorrentes - RNN}
Segundo \cite{karpathy2015visualizing}, Redes Neurais Recorrentes têm tido sucesso em aplicações que envolvam dados sequenciais (sejam simbólicos ou numéricos). Devido a dificuldade de treino desse tipo de arquitetura de rede neural, diversas propostas de melhorias foram pesquisadas. Dentre as variantes de maior sucesso, podemos destacar a \textit{Long Short Term Memory - LSTM}, proposta por \cite{lstm1997}, que, em princípio, é capaz de armazenar e recuperar informação por longos períodos de tempo.

% texto adaptado?
Entende-se, portanto, que o intuito desta pesquisa é o de reproduzir e verificar os resultados apresentados em \cite{agarwala2017music}. Além de uma discussão das ferramentas de deep learning utilizadas para composição musical, também aplicaremos as técnicas e métodos estudados para a composição de músicas inéditas baseadas em um conjunto personalizado de faixas musicais (trilhas sonoras de jogos, músicas ambientes, entre outras).

\subsection{RNN - Recurrent Neural Network}
% fonte: https://www.ibm.com/cloud/learn/recurrent-neural-networks



\subsubsection{CharRNN}
\subsubsection{LSTM - Long Short Term Memory}

% explicar a arquitetura da rede (com imagens)



\subsection{TensorFLow}
% será que é válido explicar? computação GPU, etc.



%%%%%%%%%%%%%%%%%%%%%%%%%%%%%%%%%%%
%%%%  Metodologia
%%%%%%%%%%%%%%%%%%%%%%%%%%%%%%%%%%%
\newpage

\section{Metodologia}

\subsection{Coleta dos Dados}

% resumo como foram coletados os dados (fonte também) e que isso resultou em error 403 (acesso negado) no site hehe

\subsection{Tratamento dos Dados}

% explicar a forma que tratei os arquivos .abc:
%   - removi títulos
%   - removi letras das músicas
%   - removi caracteres de comentários
%   - codifiquei/vetorizei as strings para que fosse possível ir e vir (caractere --> index e vice-versa)

\subsection{Rede Neural}

% arquiteturas de redes, parâmetros, etc.

\subsection{Pesquisa}

% avaliar percepção humana referente às músicas geradas


%%%%%%%%%%%%%%%%%%%%%%%%%%%%%%%%%%%
%%%%  Resultados
%%%%%%%%%%%%%%%%%%%%%%%%%%%%%%%%%%%
\newpage
\section{Resultados}

%%%%%%%%%%%%%%%%%%%%%%%%%%%%%%%%%%%
%%%%  Conclusão
%%%%%%%%%%%%%%%%%%%%%%%%%%%%%%%%%%%
\newpage
\section{Conclusão}

%%%%%%%%%%%%%%%%%%%%%%%%%%%%%%%%%%%
%%%%  Próximos Passos
%%%%%%%%%%%%%%%%%%%%%%%%%%%%%%%%%%%
\newpage
\section{Próximos Passos}


%%%%%%%%%%%%%%%%%%%%%%%%%%%%%%%%%%%
%%%%  Bibliografia
%%%%%%%%%%%%%%%%%%%%%%%%%%%%%%%%%%%
\newpage
\addcontentsline{toc}{chapter}{Referências Bibliográficas} % Coloca no sumário
\bibliographystyle{apalike-br}
\bibliography{biblio}


%\printindex % Opcional  Índice remissivo

\end{document}
